	\section{Introduction}
        \subsection{Purpose of Document}
		This document should provide an overview of the \project project.\\ For developer how the settings and implementation work. For user of the \project it provides detailed steps for setting up the environment.
		
        \subsection{Target Audience}
        Developer for WindowManager and normal Linux user who want to improve and lighten up their desktop/window manager.
		
		\subsection{Versioning}
		\begin{table}[H]
			\centering
			\tableformat
			\begin{tabular}{c|l}
			v1.0 & first document with requirements and design decisions\\
			\end{tabular}
			\label{version table}
			\caption{version overview}
		\end{table}
		
		\subsection{Glossary}
		\begin{table}[H]
			\centering
			\tableformat
			\begin{tabular}{l|l}
			desktop & the graphical part where the programs run\\
			clients & the running programs in your desktop\\
			window & synonym to clients\\
			HUD & heads-up display\\
			taskbar & an area where an overview of the running programs is shown\\
			RAM & Random Access Memory; Hardware of the computer\\
			CPU & Central Processing Unit; Hardware of the computer\\
			IDE & Integrated Development Environment; Collection of tools to develop software\\
			Conky & A tool to graphically display certain information such as CPU usage or RAM usage\\
			alttab & a simple script for <alt>+<tab> program switch for the Window Manager Awesome\\
			Window Manager & A software that allows managing of running Graphical User Interfaces\\
			Awesome & a lightweight window manager\\
			Cairo-Dock & A program to display various forms of taskbars\\
			Image & A piece of digital information about an operating system\\
			alsamixer & linux sound system\\
			amixer & a command line tool to handle the sound system alsamixer\\
			xbacklight & a command line tool to increase/decrease the screen brightness\\
			IP & short for Internet Protocol addresses\\
			external IP & the IP shown oustide the local network\\
			internal IP & the IP shown inside the local network\\
			\end{tabular}
			\label{glossary}
			\caption{glossary}
		\end{table}