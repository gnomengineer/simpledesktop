		\subsection{Awesome}
    			\subsubsection{Application Rules}
    				The behavior of certain applications can be done by rules in awesome. This rules look in a general manner like the code below
    				\begin{lstlisting}[caption={example ruling}]
awful.rules.rules = {
 { rule = { }, properties = { property = value, property = value , ...} },
}
    				\end{lstlisting}
    				
    				The ''rule={}'' part selects which window is affected by it. possible selectors are \textbf{rule, rule\_any, except} and \textbf{except\_any} (shortly rule selects matching, except selects non-matching and x\_any must match \textit{one} or \textit{more})\\
    				The ''property={}'' part defines the different properties for the selected window. A list of all properties and detailed explanation about rules can be found here: 
    				\url{https://awesome.naquadah.org/wiki/Understanding_Rules#Window_Properties}\\
    				
\noindent  The terminal is tiled, 50\% transparent and has no border (exchange urxvt with the desired terminal)
				\begin{lstlisting}[caption={ruling for terminals}]
{ 	rule = { name = "urxvt" }, 
	properties 
	{
		opacity = 0.5, 
		floating = false, 
		maximized = false, 
		fullscreen = false 
	}
},
				\end{lstlisting}
				
				The browser and the used IDE's have to be maximized with borders. (again, the application name can be exchanged)
				\begin{lstlisting}[caption={ruling for IDE and browser}]
{ 	rule = { class = "chromium" }, 
	properties 
	{ 
		floating = false,
		maximized = true, 
		fullscreen = false,
		border_width = beautiful.border_width,
		border_color = beautiful.border_normal 
	} 
},
				\end{lstlisting}
				
				All other applications use default border and are floating
				\begin{lstlisting}[caption={ruling for default applications}]
{ 	rule = { class = "gedit" }, 
	properties 
	{ 
		floating = true,
		maximized = false, 
		fullscreen = false,
		border_width = beautiful.border_width,
		border_color = beautiful.border_normal 
	} 
},
				\end{lstlisting}
				
			\subsubsection{Application Opening}
				to be defined\ldots
				
			\subsubsection{Beautiful Design}
				The Beautiful library defines user themes such as color, background image or icons. The variables and settings are written in a separate lua file called 'theme.lua' under /usr/share/awesome/themes/<yourtheme>/theme.lua The \project does only need the the background option of the theme.lua.\\
				The Background can be set by simple commands over the console.
				\begin{lstlisting}
 in testing
				\end{lstlisting}			
				
			\subsubsection{Keybindings}
				The \project has various key bindings to open and close applications, open pop dialogs, changing desktop. The Mapping between key and function is shown in table \ref{keymapping} in the appendix.\\
				The special keys are Mod (Windows Logo key), LShift, RShift, LControl, RControl.
				
			\subsubsection{Print Screen}
				the print screen script is executed over a shortcut (see table \ref{keymapping} for reference). The print screen will be done 2 different ways.\\
				A simple single press will take a single screenshot on the focused window.\\
				
				\begin{lstlisting}[caption={print screen command}]
awful.key({}, "PrtScr", function () awful.util.spawn("scrot -u ") end),
				\end{lstlisting}
				this code executes the shell command \textit{scrot} with the parameter \textit{-u} which takes an instant screenshot of the focused window.
			
				\shadebox{A longer press (2 seconds) will start a selection to take a screen shot of. \textbf{NOT IMPLEMENTED YET}}
			\subsubsection{Desktop Addition}
				To dynamically add desktop (or tags as they are called by awesome) the shifty library is used.\\
				With \textit{shifty} tags can be configured to be generated once a specific client opens or key-bindings can be set to add, rename, delete, swap tags.
				
				\begin{lstlisting}[caption={example of dynamic adding with shifty}]
...
awful.key({modkey}, "XF86Back", shifty.shift_prev),
awful.key({modkey}, "XF86Forward", shifty.shift_next),
awful.key({modkey}, "t", function() shifty.add({ rel_index = 1 }) end),
awful.key({modkey, "Control"},
            "t",
            function() shifty.add({ rel_index = 1, nopopup = true }) end
            ),
...
				\end{lstlisting}
				
				\shadebox{Attention! if you implement shifty without a template you have to remove everything related to tag creation and the manage functions. \href{https://awesome.naquadah.org/wiki/Shifty\#Advanced}{more details\ldots}}
			\subsubsection{Brightness}
				The brightness is handled with the \textit{xbacklight} program. it uses simple parameters for decreasing and increasing the brightness.
				\begin{lstlisting}
    awful.key({ }, "XF86MonBrightnessDown", function ()
        awful.util.spawn("xbacklight -dec 15") end),
				\end{lstlisting}
				\textit{xbacklight -dec 15} decreases the brightness via 15\%\\
				\textit{xbacklight -inc 15} increases the brightness via 15\%
				
			\subsubsection{Volume Handling}
				The volume display is handled by the conky application. Awesome is responsible for changing the percent and toggle the volume (mute/unmute).\\
				This is simply done by a key-binding to the appropriate keys.
				\begin{lstlisting}
awful.key({ mod }, "XF86AudioRaiseVolume", function ()
		awful.util.spawn("amixer set Master 9%+") end),
				\end{lstlisting}
				Using the \textit{amixer} command: whereas 9\% changes the volume by 9\% upwards if there is a + (plus) appended or downward if a - (minus) is appended. For toggling the sound on or off, the command \textit{amixer set Master \textbf{toggle}} is used.\\
				For defining the appropriate keys see \ref{keymapping}
				
			\subsubsection{Battery Warning}
				The battery usage is shown in the conky application. Nevertheless a pop-up warning is displayed on the screen to warn the user immediately when the battery is lower than 10\%\\
				The message shows up in the top right corner has a warning red background and a dark red foreground color.\\
				\begin{lstlisting}
local function trim(s)
  return s:find'^%s*$' and '' or s:match'^%s*(.*%S)'
end

local function bat_notification()
  
  local capacity_raw = assert(io.open("/sys/class/power_supply/BAT0/capacity", "r"))
  local status_raw = assert(io.open("/sys/class/power_supply/BAT0/status", "r"))
  local power_raw = assert(io.open("/sys/class/power_supply/BAT0/power_now","r"))
  local energy_raw = assert(io.open("/sys/class/power_supply/BAT0/energy_now","r"))
  
  local battery_capacity = tonumber(capacity_raw:read("*all"))
  local battery_status = trim(status_raw:read("*all"))
  local battery_usage = tonumber(power_raw:read("*all"))
  local battery_energy = tonumber(energy_raw:read("*all"))
  
  local battery_time = battery_energy / battery_usage * 60
  local battery_time_minute =  math.floor(battery_time % 60)
  local battery_time_hour = math.floor((battery_time - battery_time_minute) / 60)

  if (battery_capacity <= 10 and battery_status == "Discharging") then
    naughty.notify({ title      = "Battery Warning"
	 , text       = "Battery low! " .. battery_capacity .."%" .. " left! \n " 
	 .. battery_time_hour .. ":" ..battery_time_minute .. " remaining"
	 , fg="#ff0000"
	 , bg="#deb887"
	 , timeout    = 15
	 , position   = "top right"
    })
  end
end

battimer = timer({timeout = 120})
battimer:connect_signal("timeout", bat_notification)
battimer:start()
				\end{lstlisting}
				
			\subsubsection{Conky HUD}
				awesome is able to display Conky as an HUD and being toggled to front and back. This is done with following code
				\begin{lstlisting}
do
    local conky = nil

-- iteration through all clients to find conky
    function get_conky(default)
        if conky and conky.valid then
            return conky
        end

        conky = awful.client.iterate(function(c) return c.class == "Conky" end)()
        return conky or default
    end
    
-- raise and lower conky
    function raise_conky()
        get_conky({}).ontop = true
    end
    function lower_conky()
        get_conky({}).ontop = false
    end
    
-- creates a timer of 0.01s
    local t = gears.timer({ timeout = 0.01 })
    t:connect_signal("timeout", function()
        t:stop()
        lower_conky()
    end)
    
-- calls a timer 't'
    function lower_conky_delayed()
        t:again()
    end
    
-- toggles conky
    function toggle_conky()
        local conky = get_conky({})
        conky.ontop = not conky.ontop
    end
end
				\end{lstlisting}
				
				For the code above to work conky itself needs some awesome client rules and key-bindings  assigned.
				\begin{lstlisting}
{ rule = { class = "Conky" },
  properties = {
      floating = true,
      sticky = true,
      ontop = false,
      focusable = false
  } }
				\end{lstlisting}
				\begin{lstlisting}
awful.key({Mod}, "C", function() toggle_conky() end)
				\end{lstlisting}